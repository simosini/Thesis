% Chapter 7

\chapter{Conclusions} % Main chapter title

\label{Chapter7} % For referencing the chapter elsewhere, use \ref{Chapter7} 

%----------------------------------------------------------------------------------------
\section{Final considerations}
MiRNA's targeting is a complex, yet not fully understood, mechanism. In an ideal scenario it would be possible to experimentally verify the target set for all miRNAs, but both the cost and the limited throughput of current methods imply that miRNA studies still depend on computational predictions to complement experimental data.

On the other side, its imprecise nature makes the decision of which computational approach to employ for the task, extremely challenging. Often, one of the hardest part of solving any Machine Learning problem, can be, in fact, finding the right algorithm for the job. A certain model is better suited for different types of data and different problems and it's never easy to pick the right one. This is also due to the so called 'no free lunch theorem' \cite{nfl}, which roughly states that there is no 'perfect' Machine Learning algorithm that will perform well at any problem.

For the miRNA's targets prediction task, however, the need of finding suitable patterns to distinguish functional binding sites from non-functional, guided our choice towards a  neural network. 

Nowadays, deep learning has been exploited in many fields for various tasks, from image recognition to natural language processing with different results. The main advantage of its use consists in its capacity to automatically extract its own data feature descriptors. However, the more complex the network architecture, the more computational power (and data) it needs to properly extract valuable knowledge.

In this thesis we tackled the problem of miRNA target classification adopting a neutral approach towards the prediction process, avoiding incorporating any hand-crafted feature related to the targeting process. The obtained results and the comparison between DeepMiRNA and other descriptor-based tools suggest that current knowledge is still not sufficient to accurately capture all aspects of the miRNA targeting process. In fact DeepMiRNA's better performance over most of the available feature-based predictors, indicates that the descriptors learned by our neural network are able to encode both current knowledge and additional information yet to be determined.   

Undoubtedly, the most challenging part in DeepMiRNA's development was the processing of different data sources to build representative train, test and validation sets. For this work we selected Diana TarBase and mirTarBase as our core data sources because they represented the most comprehensive set of evidence for miRNA:mRNA functional interactions. However, for most of the validated experiments the databases do not provide exact details of the target site for the supported interactions and do not contain data about the exact transcript of the experiment but only the identification of the gene. Unfortunately, any gene is associated with a variable number of transcripts and this lack of information forced us to make the decision of selecting the one with the longest sequence, potentially inserting unwanted noise to the data.    Also, in order to  obtain reliable binding site information we integrated PAR-CLIP \cite{grosswendt} and CLASH \cite{helwak} datasets, which provided us with these information, and cross-referenced them with the main datasets. These additional data, though, turned out to be almost exclusively related to experimentally verified positive data, containing only a few examples of negative validates pairs. This, again, lead us to the decision of artificially creating negative samples to be used for training stage.  

Thus, we can affirm that the amount of validated data available for the task is still not sufficient to capture all characteristics of the targeting process; this is the reason why we incorporated a set of rules to select the best binding sites candidates to be processed by the neural network. In an ideal scenario, with enough representative positive and negative data samples, this step could be skipped as a deep enough neural network should be able to map such information into its weights. Moreover, the requirement for partial complementarity within the seed region defining the CSSM used, seems to be universally accepted and established through numerous experimental studies \cite{common_features}.

The binding sites selection steps also involved the relaxation of the canonical seed region into an extended miRNA subsequence comprising $10$ nucleotides from the first to the tenth (in the best configuration), to accommodate the presence of non-canonical targets: we called this sequence \emph{extended seed region}. This choice resulted in a better overall performance compared to other softwares utilizing a more conservative approach that considers only canonical binding sites and suggests that perfect seed region complementarity is not a sufficient discriminant to correctly identify miRNA's targets.

Another important task within this thesis was the choice of the neural network structure. Inspired by this article \cite{continuous_representation} we decided to face the vectorization step using two different algorithms: on one side the classical one-hot encoding of the sequences and on the other the continuous representation of biological sequence based on Dnad2Vec \cite{dna_distributed_repr}. This gave us the opportunity to tackle the miRNA targeting process from two different points of views.

Testing results have decreed that the classical approach using a MLP design offers a more reliable and robust classifier. This is most likely due to the lack of complete representativeness exhibited by the available data and the more complex structure of a CNN that requires a wider range of training data to correctly set its weights. 

\section{Future work}
Despite the enhanced performance demonstrated by DeepMiRNA, it is prudent to consider
some of the potential limitations of automatic feature learning approaches such as DL. The hierarchical internal data representation learned by a neural network can be sometimes be difficult to interpret and map into human interpretable knowledge, hence it is not possible to directly identify the features that determine the classification. We tried to address this issue by looking at the weights learned after the training process using specific techniques such as Grad-CAM \cite{gradcam} and other visualization methods such as the ones described in this article \cite{nlp_visualizing}, but unfortunately we were not able to extract any valuable information. It's worth mentioning, though, that further investigations in that direction may aid the interpretation process and improve the classification accuracy, which is the next logical step in our work.

Another important improvement to DeepMiRNA could concern the extension to the whole gene transcript rather than only analyzing the 3' untranslated region. Recent studies \cite{mirwalk} \cite{helwak}, in fact, revealed the importance of the whole sequence, comprising both coding and 5' untranslated regions, in miRNA's targets prediction. Although over 60\% of miRNA:mRNA interactions take place inside the 3'UTR, we believe that extending the search of potential candidate sites to other regions is most likely to improve the accuracy of the predictor. 

For the development of DeepMiRNA, anyhow, we decided to only use data relative to the 3'UTR because it has still not been fully clarified if actual bindings to other regions exhibit the same potential to repress mRNA translation. 

Besides, even though the work presented in this thesis focused on the prediction of human miRNA targets, the same methodology can be applied to build target prediction models for any other living species, aware that gathering enough representative data will be of crucial importance for the new task. 

We conclude this thesis by considering that the presented approach will certainly benefit from further experimental studies that will serve to validate new predictions obtained by DeepMiRNA, but also to generate new experimental data to reliably expand the training of the model in both its configurations.
 

  