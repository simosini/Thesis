% Chapter 6

\chapter{Conclusions} % Main chapter title

\label{Chapter6} % For referencing the chapter elsewhere, use \ref{Chapter6} 

%----------------------------------------------------------------------------------------

MiRNA's targeting is a complex, yet not fully understood, mechanism. In an ideal scenario it would be possible to experimentally verify the target set for all miRNAs, but both the cost and the limited throughput of current methods imply that miRNA studies still depend on computational predictions to complement experimental data.

On the other side, its imprecise nature makes the decision of which computational approach to employ for the task, extremely challenging. Often, one of the hardest part of solving any Machine Learning problem, can be, in fact, finding the right algorithm for the job. Different models are better suited for different types of data and problems and it's never easy to pick the right one. This is also due to the so called 'no free lunch theorem' \cite{nfl}, which roughly states that there is no 'perfect' Machine Learning algorithm that will perform well at any problem.

For the miRNA's targets prediction task, however, the need of finding suitable patterns to distinguish functional binding sites from non-functional, guided our choice towards a  neural network. 


  